\documentclass[11pt,twoside]{handout}
\usepackage{programs1}

\author{P. N. Hilfinger}
\title{The GJDB Debugger}
\usepackage{programs1}

\begin{document}
\makecompacttitle

A {\it debugger\/} is a program that runs other programs, allowing its
user to exercise
some degree of control over these programs, and to examine them when
things go amiss.  
Sun Microsystems, Inc.\ distributes a  text-based
debugger, called JDB, with its Java Developer's Kit
(JDK).    I have
modified JDB to make its commands look pretty much like GDB, the GNU
Debugger\footnote{The recursive acronym GNU means ``GNU's Not Unix''
and refers to a larger project to provide free software tools.},
which handles C, C++, Pascal, Ada, and a number of 
other languages.  The result is called GJDB (g'jay dee
bee)\footnote{This document applies to GJDB version .}.  Perhaps
the most convenient way to use it is through the interface supplied
with Emacs.  

GJDB is dauntingly chock-full of useful stuff, but for our purposes, a
small set of its features will suffice.  This document describes them.

\section{Basic functions of a debugger}

When you are executing a program containing errors that manifest
themselves during execution, there are several things you might want
to do or know.
\begin{itemize}
\item What statement or expression was the program executing at the
time of a fatal error? 
\item If a fatal error occurs while executing a function, what line of
the program contains the call to that function?
\item What are the values of program variables (including parameters)
at a particular point during execution of the program?
\item What is the result of evaluating a particular expression at some
point in the program?
\item What is the sequence of statements actually executed in a
program?
\item When does the value of a particular variable change?
\end{itemize}
These functions require that the user of a debugger be able to {\it examine\/}
program data, to obtain a {\it traceback\/} (a list of function calls
that are currently executing sorted by who called whom), to set  {\it
breakpoints\/} where execution of the program is suspended to allow
its data to be examined, and to {\it step\/} through the statements of
a program to see what actually happens.  GJDB provides all these
functions.  It is a {\it symbolic\/} or {\it source-level\/} debugger,
creating the fiction that you are executing the Java
statements in your source program rather than the machine code they have
actually been translated into.

\section{Preparation}
A typical Java implementation first compiles (translates) Java
programs into executable files containing {\it bytecode,\/} a sort of
machine language for an idealized virtual machine that is 
considerably easier to execute than the original source text.
This translation process
generally loses 
information about the original Java statements that were translated.  A
single Java statement usually translates to several machine statements,
and most local variable names are simply eliminated. Information about
actual variable
names and about the original Java statements in your source program is
unnecessary for simply executing your program.
Therefore, for a source-level debugger to work properly, the compiler
must retain some of this superfluous information
(superfluous, that is, for execution).  

To indicate to our compiler ({\tt javac}) that you intend to debug your
program, and therefore need this extra information, add the \prog|-g|
switch during both compilation.  For example, if you are compiling an 
application whose main class is called {\tt Main}, you
might compile with 
\begin{program}
javac -g Main.java
\end{program}

This  sample command sequence  produces a {\it class file\/}
\prog|Main.class| containing the translation of the class {\tt Main},
and possibly some other class files.  

\section{Starting GJDB}
To run this under control of \prog|gjdb|, you
can type
\begin{program}
gjdb Main
\end{program}
in a shell.  You will be rewarded with the initial command prompt:
\begin{program}
[-] 
\end{program}
This provides an effective, but unfrilly text interface to
the debugger.  I don't actually recommend that you do this; it's
much better to use the Emacs facilities described below.  However, the
text interface will do for describing the commands.

\section{Threads and Frames}\label{sec:frames}
When GJDB starts, your program has not started; it won't until
you tell GJDB to run it (you tell the program is not started from 
GJDB's prompt, which will be {\tt[-]}). 
After the program has started and before it exits, GJDB will see a set
of {\it threads,\/} each one of which is essentially a
semi-independent program.   If you haven't encountered Java threads
before, the part of your program that you usually think of as ``the
program'' will be the 
{\it main thread,\/} appropriately named {\tt main}.
However, there will also be a bunch of 
{\it system threads} (running various support activities), that GJDB
will tell you about if asked, but which will generally not be of interest.
GJDB can examine one thread at a time; which one being indicated by the
prompt:
\begin{description}
\item[{[-]}] Means there are no threads; the program has not been
started.
\item[{[?]}] Means the program is started, but GJDB is not looking at
any particular thread.  You'll may see this if you interrupt your program.
\item[{{\it name\/}[$n\/$]}] Means that GJDB is looking at thread
{\it name\/}, and at frame \#$n\/$ (see below) within that
thread.
\end{description}

At any given time, a particular thread is in the process of executing
some statement inside a function (method)\footnote{Even when your
program is initializing a field in a record, which doesn't {\it
look\/} as if it's inside a method, it is actually executing a part of
either a constructor or a special ``static initializer'' method (which
you'll see in certain listings under the name {\tt <clinit>}).}.  To
arrive inside that method, the program had to execute a method call in
a statement of some other method (or possibly the same, in the case of
recursion), and so on back to the mysterious system magic that started
it all.  In other words, in each thread, there is a sequence of
currently active method calls, each of which is executing a particular
statement, and each of which also has a bunch of other associated
information: parameter values, local variable values and so forth.  We
refer to each of these active calls as {\it frames,\/} or sometimes
{\it stack frames,\/} because they come and go in last-in-first-out
order, like a stack data structure.  Each has a {\it current
location,\/} which is a statement or piece of a statement that is
currently being executed in that call (sometimes called a {\it program
counter\/} or, confusingly, {\it PC}).  The most recent, or {\it
top\/} frame is the one that is executing ``the next statement in the
program,'' while each of the other frames is executing a (so-far incomplete)
method call.  

For example, consider the simple class {\tt Example} on page~\pageref{fig:Example}.
Suppose we start the program with command-line argument 5, and
are stopped at statement (E).  Then (for the main thread) GJDB
sees frames \#0--\#5, as follows:

\medskip
\begin{tabular}{r | l | c | l}
\multicolumn{1}{c|}{Frame\#} & 
\multicolumn{1}{c|}{Method} & 
Location &
\multicolumn{1}{c}{Variables} \\
\hline
0. & {\tt report} & (E) & {\tt x: 2} \\
1. & {\tt ilog} & (C) & {\tt x: 1, a: 2} \\
2. & {\tt ilog} & (D) & {\tt x: 2, a: 1} \\
3. & {\tt ilog} & (D) & {\tt x: 5, a: 0} \\
4. & {\tt process} & (B) & {\tt x: "5"} \\
5. & {\tt main} & (A) & \verb|args: { "5" }|
\end{tabular}


\begin{figure}[t]
\begin{program}
class Example {
  public static void main (String[] args) {
    for (int i = 0; i < args.length; i += 1)
      process (args[i]);           // (A)
  }

  void process (String x) {
    ilog (Integer.parseInt(x), 0); // (B)
  }

  void ilog (int x, int a) {
    if (x <= 1)
       report (a);	           // (C)
    else
       ilog (x/2, a+1);		   // (D)
  }

  int report (int x) {
    System.out.println (x);        // (E)
  }
}
\end{program}
\caption{A sample program to be debugged.  Assume that it is stored in
    file {\tt Example.java}.}
\label{fig:Example}
\end{figure}

\section{GJDB Commands}\label{sec:commands}
This section describes the commands available under GJDB. Except where
noted, one
uses them when the program being debugged by GJDB is stopped.
You can abbreviate
most commands with a sufficiently long prefix.  For example, {\tt p} is
short for {\tt print}, and {\tt b} is short for {\tt break}.

\subsection{Basic Commands}
The following basic commands give you enough to pinpoint where your
program blows up, and usually to find the offending bad pointer or
array index that is the immediate cause of the problem (of course, the
actual error probably occurred much earlier in the program; that's why
debugging is not completely automatic.)  Personally, I usually don't
need more than this; once I know where my program goes wrong, I often
have enough clues to narrow down my search for the error.  You should
{\it at least\/} establish the place of a catastrophic error before
seeking someone else's assistance. 

\begin{description}
\item[\tt help {\it command}] \ \\
Provide a brief description of a GJDB command or topic.  Plain {\tt
help} lists the possible topics.
\item[\tt run {\it command-line-arguments}] \ \\
Starts your program as if you had typed
\begin{program}
java \{\it Main\} \{\it command-line-arguments\}
\end{program}
to a Unix shell.  GJDB remembers the arguments you pass, and plain {\tt
run} thereafter will restart your program from the top with those
arguments.  By default, the standard input to your program will come
from the terminal (which causes some conflict with entering debugging
commands: see below).  However, you may take the standard input from
an arbitrary file by using input redirection: adding {\tt < {\it filename\/}}
to the end of the {\it command-line-arguments\/} uses the contents of
the named file as the standard input (as it does for the shell).  
Likewise, adding {\tt> {\it filename\/}} causes the standard output
from your program to go to the named file rather than to the terminal,
and
{\tt >\& {\it filename\/}} causes both the standard output and the 
standard error output to go to the named file.
\item[\tt where] \ \\
Produce a backtrace---the chain of function calls that brought the
program to its current place.  The commands {\tt bt} and {\tt
backtrace} are synonyms.
\item[\tt up] \ \\
Move the current frame that GJDB is examining to the caller of that
frame.  Very often, your program will blow up in a library
function---one for which there is no source code available, such as
one of the I/O routines.  You will need to do several {\tt up}s to get
to the last point in your program that was actually executing.
Emacs (see below) provides the shorthand {\tt C-c<} (Control-C
followed by less-than), or the function key {\tt f3}.
\item[\tt up $n\/$] Perform $n\/$ {\tt up} commands ($n$ a positive number).
\item[\tt down] \ \\
Undoes the effect of one {\tt up}.  Emacs provides the shorthands {\tt
C-c>}
and function key {\tt f4}.
\item[\tt down $n\/$] Perform $n\/$ {\tt down} 
	  commands ($n$ a positive number).
\item[\tt frame $n\/$] Perform {\tt up}s or {\tt downs}s as needed to
make frame \#$n\/$ the current frame.
\item[\tt thread $T\/$] Make thread $T\/$ the current thread that GJDB 
is examining.  $T\/$ may either be the name or the number of a thread
(as reported by {\tt info
threads}, below).
\item[\tt print $E$] \ \\
prints the value of $E$ in the current frame in the program, where $E$
is a Java expression (often just a variable).  
For example
\begin{program}
main[0] print A[i]
A[i] = -14
main[0] print A[i]+x
A[i]+Main.x = 17
\end{program}
This tells us that the value of \prog|A[i]| in the current frame is -14
and that when this value is added to \prog|Main.x|, it gives 17.
Printing a reference value is less informative:
\begin{program}
main[0] p args
args = instance of java.lang.String[3] (id=172)
\end{program}
This tells you that {\tt args} contains a pointer to a 3-element array
of strings, but not what these strings are.  
\item[\tt print/$n~E$] 
also prints the value of expression $E$ in the current frame. If
$E$ is a reference value, however, it also prints the subcomponents
(fields or array elements) of the
referenced object to $n$ levels.  Plain {\tt print} without this
specification is equivalent to {\tt print/0}, and does not print
subcomponents. {\it Printing subcomponents
to one level\/} means printing each subcomponent of $E$'s
value as if by {\tt print/0}.  Printing to two levels prints means
printing each subcomponent as if by {\tt print/1}, and so forth recursively.
For example,
\begin{program}
main[0] print/1 args
args = instance of java.lang.String[3] (id=172) {
"A", "B", "C"
}
main[0] p T
T = instance of Tree(id=176)
main[0] p/1 T
T = instance of Tree(id=176) {
  label: "A"
  left: null
  right: instance of Tree(id=178)
}
main[0] p/2 T
T = instance of Tree(id=176) {
  label: "A"
  left: null
  right: instance of Tree(id=178) {
    label: "B"
    left: null
    right: instance of Tree(id=180)
  }
}
\end{program}
\item[\tt dump $E$] \ \\
Equivalent to {\tt print/1 $E$}.  
\item[\tt dump/$n~E$] \ \\
Equivalent to {\tt print/$n~E$}.  
\item[\tt call $E$] \ \\
Evaluate $E\/$ for its side effects, without printing the result.
\item[\tt info locals] Print the values of all parameters and 
local variables in the current frame.
\item[\tt info threads] List all current threads.
\item[\tt quit] \ \\
Leave GJDB.
\end{description}

\subsection{More Advanced Commands}

The next bunch of commands allow you to actively stop a program during
normal operation.
\begin{description}
\item[{\tt suspend} and {\tt C-f}] ~\\
When a program is run from a Unix shell,  {\tt C-c} will terminate
its execution (usually).  At the moment, unfortunately, it will
also do this to GJDB itself.  When debugging, you usually want instead
to simply stop the debugged program temporarily in order to examine it. 
When the standard input is redirected from a file (using `{\tt<}'; see 
the {\tt run} command), you can simply use {\tt suspend} to stop the
program (and then use {\tt continue} or {\tt resume} to restart).  
When the program is running and standard input comes from the
terminal, things get complicated: how does GJDB know a command from
program input.  If you are using GJDB mode (see \S\ref{sec:emacs}),
then {\tt C-c C-c} will do the trick in this case.
Otherwise, if you are running in an ordinary shell, use {\tt C-f}
following by return.  And finally, if you are running in a shell under
Emacs, use {\tt C-qC-f} followed by return.
\item[\tt break {\it place}] \ \\
Establishes a breakpoint; the program will halt when it gets there.
The easiest breakpoints to set are at the beginnings of functions, as
in
\begin{program}
[-] break Example.process
Set breakpoint request Example:8
\end{program}
(using the class {\tt Example} from \S\ref{sec:frames}).  Use the
full method name (complete with class and package qualification), as
shown.  You will
either get a confirming message as above (saying that the system set a
breakpoint at line~8 of the file containing class {\tt Example}),
or something like 
\begin{program}
Deferring BP RatioCalc.main [unresolved].
\end{program}
when you set a breakpoint before the class in question has been
loaded.  
This means that the breakpoint will be set when (and if) the class in which
it occurs is loaded.
A {\it place\/} may also indicate a line number in a class, 
so that you may break
on any line of a program.  For example, to place a breakpoint at
line~13  (point (C)) of the program
in Figure~\ref{fig:Example}, type
\begin{program}
[-] b Example:13
\end{program}
(and as you can see, I have abbreviated `{\tt break}' as `{\tt b}'
just to show it's legal).
Emacs allows you to set breakpoints of this sort 
with the mouse (see \S\ref{sec:emacs}).

Breakpoints in anonymous classes are a bit tricky; their names
generally have the form ``{\tt$C$\$$n$}'' where $C$ is the name of the
outermost class enclosing them, and $n$ is some integer.  The problem
is that you don't generally know the value of $n$.  GJDB therefore
allows ``{\tt$C$.0}'' as a class name, meaning ``any anonymous class
inside $C$.''

When you run your program and it hits a breakpoint, you'll get a
message and prompt like this:
\begin{program}
Breakpoint hit: thread="main", Example.main(), line=4, bci=22
main[0] 
\end{program}
(Here, ``bci'' indicates a position within the bytecode translation of
the method; it is not generally very useful).
\item[{\tt command} $N$]
    Add (or delete) commands to breakpoint number $N$ that will be
    executed whenever the breakpoint is hit (assuming any conditions on the 
    breakpoint are satisfied).  Prompts for commands to be typed in on
    subsequent lines, ending with a line containing just the word {\tt
    end.}  To delete commands, simply make this trailing {\tt end} be the
    only line.
\item[{\tt command}] As for `{\tt command $N$}', above, using the
	   latest breakpoint set as $N$.	
\item[{\tt condition} $N$ {\it cond\/}] Make breakpoint number $N$ 
	   conditional, so that the program only stops if {\it cond,\/} which
	   must be a boolean expression, evaluates to true.
\item[{\tt condition} $N$] Make breakpoint number $N$ unconditional. 
\item[{\tt clear} {\it place\/}\dots] Undoes one or more {\tt break}s.
You may also use breakpoint numbers for {\it place}.
\item[\tt delete] \ \\
Removes breakpoints, watchpoints, and catches by number.
This form of the command gives you a choice of 
breakpoints to delete, and is generally most convenient.
\item[\tt ignore $I\/$ $N\/$] Cause the breakpoint, watchpoint,
or catch numbered $I\/$ to be ignored the next $N\/$ times it is hit.
\item[\tt info break] Lists current breakpoints and any conditions or
commands on them.
\item[\tt info watch] Lists current watchpoints (as set by the {\tt
watch} command).
\item[\tt info catch] Lists current exceptions intercepted by GJDB, as
set by the {\tt catch} command.
\item[\rm{\tt cont} or {\tt continue}]  \ \\
Continues regular execution of  the program from a breakpoint or other stop.
\item[\tt step] \ \\
Executes the current line of the program and stops on the next statement to
be executed.  
\item[\tt next] \ \\
Like {\tt step}, except that if the current line of the program contains a
function call (so that {\tt step} would stop at the beginning of that
function), does not stop in that function.    
\item[\tt finish] \ \\
Does {\tt next}s, without stopping, until the 
current method (frame) exits.
\item[{\tt watch} {\it P.field}] Stop the program whenever
the indicated field is assigned to by the program. {\it P\/} may either
be the name of a class, or an expression in parentheses.  If {\it P\/}
is a class name, then all assignments to the given field (of any
instance of {\it P\/}) are caught.  If {\it P\/} has the form
{\tt($E\/$)}, where $E\/$ is a non-null reference-valued expression,
then $E\/$ is evaluated to yield a reference to an object
and only assignments to the indicated field of that particular object
are watched (however, if the indicated field is static, then all
assignments to the field are watched).  It is the value of $E\/$ at
the time the {\tt watch} command is entered that matters.    For
example, after `{\tt watch (L).next},' changes to the variable {\tt L}
will have no effect on which object is observed.
\item[{\tt watch access} {\it P.field}] Stop the program whenever
the indicated field is read by the program (that is, its value is used).
\item[{\tt watch all} {\it P.field}] Stop the program whenever
the indicated field is read or assigned to by the program.
\item[{\tt unwatch}\dots] Same arguments as {\tt watch}, but removes
the watchpoints set by the corresponding {\tt watch} command.
\item[{\tt unwatch}] Remove watchpoints. Provides a choice of which to
remove, as for {\tt delete}.
\item[{\tt catch} {\it class\/}] Stop the program at the point where
an exception of type {\it class\/} is thrown. GJDB will
stop automatically at the throw of any exception that will not be caught.
It will not normally stop on exceptions that are caught by the
program.
The {\tt catch} command directs GJDB to stop on the throw of {\it
class\/} even if it will be caught.  The {\it class\/} should be fully
qualified, as in {\tt java.io.IOException}.
\item[{\tt pass}] Stop catching exceptions.  This form of the
command gives you a choice of exceptions to ignore.
\end{description}

\subsection{Attaching to Running Programs}
The {\tt run} command creates a new Java virtual machine and executes
a target program from the beginning.  You can also use GJDB on Java
programs that are already running, if they are prepared to accept
control by a debugger.  To start a program, {\tt Foo}, so that GJDB
can gain control of it, use a command such as
\begin{program}
java -agentlib:jdwp=transport=dt_socket,server=y Foo
\end{program}
This starts program {\tt Foo} and print out a message such as
\begin{program}
Listening for transport dt_socket at address: 13901
\end{program}
You can use the address 13901 within GJDB to {\it attach\/} to the
running program.  It is also possible to set things up so that the
program goes looking for a waiting instance of GJDB:
\begin{program}
java -agentlib:jdwp=transport=dt_socket,address=13000 Foo
\end{program}
where previously you had told GJDB to listen at address (socket)
13000.  

The relevant commands in GJDB are as follows:
\begin{description}
\item[\tt attach {\it N\/}] Attach to the running process at address
$N$.
\item[{\tt listen [{\it N\/}]}] Wait for a Java process to attach on
port $N$.  When $N$ is defaulted, GJDB selects a port and prints its
number.
\item[\tt connect {\it SPEC\/}] Connect to the Java VM specified by
{\it SPEC}.  This is the most general form of connection, for use by
experts.
\item[detach]  Releases the Java VM currently being debugged.  If this
VM was attached or listened to, it can continue execution.  If it was 
created by {\tt run}, it will terminate.
\end{description}

For further details on these topics, see the Sun documentation on 
``Connection and Invocation Details'' in the on-line documentation for
``Java Platform Debugger Architecture''.


\section{Common Problems}
\paragraph{Name unknown.} When you see responses like this:
\begin{program}
main[0] print x
Name unknown: x
main[0] print f(3)
Name unknown: f
\end{program}
check to see if the variable or method in question is static.  A
current limitation of the debugger is that you must fully qualify such
names with the class that defines them, as in 
\begin{program}
main[0] print Example.f(3)
\end{program}
Beware also that fully qualified names include the package name.

\paragraph{Ignoring breakpoints.} For a variety of reasons, it is
possible for a program to miss a breakpoint that you thought you had
set.  Unfortunately, GJDB is not terribly good at the moment at
catching certain errors. In particular, it will tell you that a breakpoint has been deferred, when in
fact it will never be hit due to a class name being misspelled.

\section{GJDB use in Emacs}\label{sec:emacs}
While one {\it can\/} use {\tt gjdb} from a shell, nobody in his right
mind would want to do so.  {\tt Emacs} provides a much better
interface that saves an enormous amount of typing, mouse-moving, and
general confusion.  Executing the Emacs command {\tt M-x~gjdb} starts
up a new window  running {\tt gjdb}, and enables a number of Emacs
shortcuts, as well as providing a {\bf Debug} menu for issuing many 
GJDB commands.    This command prompts for a command string (typically
{\tt gjdb {\it classname\/}}) and (for certain historical reasons)
creates a buffer 
named {\tt *gud-{\it classname\/}*}.
Emacs intercepts 
output from {\tt gjdb} and interprets it for you.  When you stop at a
breakpoint, Emacs will take the file and line number reported by {\tt
gjdb}, and display the file contents, with the point of the breakpoint
(or error) marked.  As you step through a program, likewise, Emacs will
follow your progress in the source file.  Other commands allow you to
set or delete breakpoints at positions indicated by the mouse.  

The
following table describes the available commands.  On the left, you'll
find the text command line, as described in \S\ref{sec:commands}.  
Next comes the {\bf Debug} menu button (if any) that invokes the
command.  This menu applies both to the GJDB buffer and to buffers
containing {\tt.java} files.
Next come the Emacs shortcuts: sequences of keys that run the
commands.  The shortcuts are slightly different in the GJDB buffer as
opposed to buffers containing source ({\tt.java}) files, so there are
two columns of shortcuts.  The last column contains further description.
Finally, here are a few reminders about Emacs terminology:

\def\Menu#1{{\bf\footnotesize #1}}
\def\In#1{{\tt #1}}
\def\Key#1{{\tt #1}}

\begin{itemize}
\item[1.] In shortcuts, \Key{C-$x\/$} means ``control-$x$,''
\Key{S-$x\/$} means ``shift-$x$'' and \Key{f$n\/$} refers
to one of the function keys (typically above the keyboard).
\item[2.] The {\it point,\/} in Emacs, refers to the location of the
cursor; there is one for each buffer.  
You can set the point using the usual motion commands when in the
buffer, or by simply clicking the mouse at the desired spot.
\item[3.] The {\it region\/} in any given buffer is a section of text
(usually shadowed or highlighted so that you can tell where it is).
One convenient way to set it is by dragging 
the mouse over the text you want included while holding down the left
mouse button.
\end{itemize}

\newpage

\begin{table}
\caption{Summary of Commands for Program Control}

\vspace{2ex}

\hspace*{-0.5in}{\small
\begin{tabular}{@{} l | l | l | l | p{2.5in}}
  &
\multicolumn{3}{c|}{ Emacs \strut} & \\
\multicolumn{1}{c|}{Command Line}  &
\multicolumn{1}{c|}{Menu} &
\begin{minipage}{0.5in}\begin{center}GJDB \\buffer\strut \end{center}\end{minipage} &
\begin{minipage}{0.5in}\begin{center}{\tt.java} \\buffer\strut\end{center} \end{minipage} &
\multicolumn{1}{c}{Description} \\
\hline
\In{next} & \Menu{Step Over} & 
	  \begin{minipage}[t]{0.5in}\Key{f6} or \\ \Key{C-c~C-n}\strut \end{minipage}
	  & \Key{f6} &
	  Execute to the next statement of the program;  if this
	  statement contains function calls, execute them completely
	  before stopping. [See Note~3, below]\\
\In{step} & \Menu{Step Into} & 
	  \begin{minipage}[t]{0.6in}\Key{f5}, or\\ 
				     \Key{C-c C-s}\strut \end{minipage}& \Key{f5} &
	  Execute to the next statement of the program;  if this
	  statement calls a function, stop at its first line.  [See Note~3, below]\\
\In{finish} & \Menu{Finish Function} & 
	    	  \begin{minipage}[t]{0.6in} \Key{f7} or \\ \Key{C-c C-f}\strut 
		  \end{minipage}
		  & \Key{f7} &
	  Execute until the current function call returns. \\
\In{continue} & \Menu{Continue} & \begin{minipage}[t]{0.6in}\Key{f8} or\\ 
				     \Key{C-c C-r}\strut \end{minipage}& \Key{f8} &
	  Continue execution of stopped program. \\
\hline
\In{suspend} & \Menu{Interrupt} & \Key{C-c C-c} & &
	  Interrupt execution of program and suspend its threads. \\
\Key{C-f} & \Menu{Interrupt} & \Key{C-c C-c} & &
	  Same as {\tt suspend}, but works in cases where the debugged
	  program is running and GJDB is
	  passing
	  input to it from the terminal. \\
\hline
\In{break {\it class:line\#\/}} & \Menu{Set Breakpoint} &  
	  &  {\tt C-x\verb*| |}
	  & Set a breakpoint at the point (in an Emacs source buffer)
	  or at the indicated class and line number (command line). \\
\In{clear {\it class:line\#\/}} & \Menu{Clear Breakpoint} &  & 
	  & Remove a breakpoint at the point (in an Emacs source
	  buffer) or at the indicated class and line number. \\
\In{delete} & & & & Delete selected breakpoints, watchpoints, catches. \\
\hline
\In{catch {\it class\/}} & & & & Set program to halt whenever
			   exception {\it class\/} is thrown. \\
\In{pass} & & & & Stop catching selected exceptions. \\
\In{ignore $I~N\/$} & & & & Ignore breakpoint, watchpoint or catch
\#$I\/$ the next $N\/$ times it is hit. \\
\hline
\begin{minipage}{1.6in}
\begin{math} \left.
\begin{array}{@{}l}
\mbox{\In{watch $C.f$}} \\
\mbox{\In{watch access $C.f$}} \\
\mbox{\In{watch all $C.f$}} \\
\mbox{\In{watch ($E\/$).$f\/$}} \\
\mbox{\In{watch access ($E\/$).$f\/$}} \\
\mbox{\In{watch all ($E\/$).$f\/$}}
\end{array} \right\}\end{math}
\end{minipage}
& & & & 
\begin{minipage}{2.5in}
\strut The first three forms set program to halt when (any
instance of) field $f\/$ of class $C\/$ is accessed.  Without a
modifier, only assignments to $f\/$ are tracked.  With the modifier
{\tt access}, only uses of $f\/$'s value are tracked.  The
modifier {\tt all} tracks both types of access.   The last three forms
are the same, but apply only to the object referenced by $E\/$ at the
time the command is executed.\strut
\end{minipage}
\\
\strut\In{unwatch} & & & & \strut Delete selected watchpoints. \\
\hline
\In{run}  & \Menu{Run} & & &
	  (Re)start the program, using the last set of command-line arguments.
	  \\
-         & \Menu{Start Debugger} & & &
	  Run {\tt gjdb} on the class in this buffer. \\
\In{quit} & \Menu{Quit} & & &
	  Leave GJDB.
\end{tabular}
}
\end{table}

\begin{table}
\caption{Summary of Commands for Examining a Program}

\vspace{2ex}

{\small
\begin{tabular}{@{} l | l | l | l | p{2.5in}}
  &
\multicolumn{3}{c|}{ Emacs \strut} & \\
\multicolumn{1}{c|}{Command Line}  &
\multicolumn{1}{c|}{Menu} &
\begin{minipage}{0.5in}\begin{center}GJDB \\buffer\strut \end{center}\end{minipage} &
\begin{minipage}{0.5in}\begin{center}{\tt.java} \\buffer\strut\end{center} \end{minipage} &
\multicolumn{1}{c}{Description} \\
\hline
\In{print {\it expr}} & \Menu{Print} & \Key{f9}
	  & \Key{f9}  & Evaluate {\it expr\/} and print, without showing
	       any subcomponents of the value.   Emacs commands apply
	       either to the contents of the region, or if it is
	       inactive, to the variable or field selection
	       at or after the point. \\
\In{dump {\it expr}} & \Menu{Print Details} &  \Key{S-f9}
	  & \Key{S-f9} & Evaluate {\it expr\/} and print, also printing
	       any components (array elements or fields).  With Emacs,
	       gets the expression to print as for \In{print}. \\
\hline
\In{up}   & \Menu{View Caller} & 
          \begin{minipage}[t]{0.5in}\Key{f3} or \\ \Key{C-c <}\strut \end{minipage}
          & \Key{f3} &
	  Move the debugger's current focus of attention up one
	  frame; if looking at frame $n$ at the moment, we switch to 
	  frame $n-1$.  \\
\In{down}   & \Menu{View Callee} 
          & \begin{minipage}[t]{0.5in}\Key{f4} or\\ \Key{C-c >} \strut \end{minipage}
          & \Key{f4} &
	  Move the debugger's current focus of attention down one
	  frame (from frame~$n$ to frame~$n+1$).  Opposite of \In{up}. \\
\In{where} &  & & & Print a backtrace, showing all active
			    subprogram calls. \\
\In{thread $T$} & & & & Make thread $T\/$ be the one that GJDB is 
		      currently examining, where $T\/$ is a thread
		      name or thread number. \\
\hline
\In{info locals} & & & & Print (as for the {\tt print} command) the
	 values of all local variables in the current frame. \\
\In{info threads} & & & & List all threads in the program. \\
\In{info watch} & & & & List current watchpoints. \\
\In{info break} & & & & List current breakpoints. \\
\In{info catch} & & & & List current exceptions intercepted.  \\
\hline
-	  & \Menu{Refresh} & & &
	  Re-arrange Emacs' windows as needed to display the current
	  source line that GJDB is looking at.
\end{tabular}
}
\end{table}
\end{document}

